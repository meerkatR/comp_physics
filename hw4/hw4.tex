\documentclass{article}
\usepackage{epsf,epsfig,graphicx}
\renewcommand{\labelenumi}{\roman{enumi}.}




\title{HW4 Extrapolation, Pad\'{e}, and Continued Fractions}
\author{Fei Ding}

\begin{document}
\maketitle

\textbf{Problem1}

Using the transformation from continued fraction to Pad\'{e}, we can see that here $a_1=1,   a_n=(n-1)^2  \;\;(n\not=1);  bn=2n-1.$ So 
\begin{eqnarray*}
A_1&=&b_0A_0+a_0A_{-1}=A_0+A_{-1}\\
A_n&=&b_nA_{n-1}+a_nA_{n-2}=(2n-1)A_{n-1}+(n-1)^2A_{n-2}  \;\;\; (n\not=1)\\
\\
B_1&=&b_0B_0+a_0B_{-1}=B_0+B_{-1}\\
B_n&=&b_nB_{n-1}+a_nB_{n-2}=(2n-1)B_{n-1}+(n-1)^2B_{n-2}   \;\;\;(n\not=1)
\end{eqnarray*}

The code:
\begin{verbatim}
program quarterPi
integer::N
double precision:: A,A_1,A_2,B,B_1,B_2,pi,Pade,error
A_1=0d0
A_2=1d0
B_1=1d0
B_2=0d0
pi = 4.d0*datan(1.d0)
do N=1, 30
   if (N==1) then 
      A=A_1+A_2
      B=B_1+B_2
   else
      A=(2*N-1)*A_1+(N-1)**2*A_2
      B=(2*N-1)*B_1+(N-1)**2*B_2
   end if
   A_2=A_1
   A_1=A
   B_2=B_1
   B_1=B
   Pade=A/B
   Error=Pade-pi/4.d0
   write(1,*) N,Pade*4.d0, pi, Error  
end do
end program quarterPi 
\end{verbatim}

\begin{figure} [ht]
\includegraphics[width=1.0\textwidth]{4_1}
\end{figure}

The plot shows that while $N$ increases, the error decreases.   As $N$ reaches 30, the error almost hits the machine accuracy.  So the error kind of reaches some degree and doesn't decrease anymore.

\vskip 1cm




\textbf{Problem2}

\begin{enumerate}
\item expanding $\psi$ in Chebyshev polynomials of the second kind.

\[\psi(x)=\sum_{n=0}^{\infty}a_nU_n(x)\]
where $U_n(x)$ are the second kind Chebyshev polynomials satisfying properties as following,
\[U_{n+1}(x)=2xU_n(x)-U_{n-1}(x)\]
\[\int_{-1}^1\sqrt{1-x^2}U_nU_m\,dx=\frac{\pi}{2}\delta_{nm}\]
\[(1-x^2)U_n''-3xU_n'+n(n+2)U_n=0.\]

substitute the expansion of $\psi$ into the eigenvalue problem,
\[(x^2-1)\psi''+3x\psi'+\beta^2(2-2x-\frac{aE}{\beta})\psi=0\]
we get (set $a=\beta=1$)
\[\sum_{n=0}^{\infty}a_n[(x^2-1)U_n''+3xU_n'+(2-2x-E)U_n]=0\],
apply the third property of $U_n, (1-x^2)U_n''-3xU_n'+n(n+2)U_n=0$, we get
\[\sum_{n=0}^{\infty}a_n[n(n+2)+2-2x-E]U_n=0\]
apply the first property of $U_n, U_{n+1}(x)=2xU_n(x)-U_{n-1}(x)$, we get
\[a_0(2-2x-E)U_0+\sum_{n=1}^{\infty}a_n[(n^2+2n+2-E)U_n-U_{n-1}(x)-U_{n+1}(x)]=0\]
Since we know the detailed form of $U_0$ and $U_1$, $U_0(x)=1, U_1(x)=2x$, the first term can be rewritten as $a_0[(2-E)U_0-U_1]$, then apply the second property of $U_n, \int_{-1}^1\sqrt{1-x^2}U_nU_m\,dx=\frac{\pi}{2}\delta_{nm}$, that is multiplying the equation by $U_m\sqrt{1-x^2}$ and integrate from -1 to 1, we get a set of equations of coefficients $a_m$,
\begin{eqnarray*}
(2-E)a_0-a_1=0, &m=0&\\
(m^2+2m+2-E)a_m-a_{m-1}-a_{m+1}=0, &m\geq1&
\end{eqnarray*}
\item establishing a tri-diagonal form for the ode
Rearrange the equations and write the index as $n$ instead of $m$, we get,

\[\alpha_0a_0+\beta_0a_1=Ea_0\]
\[\vdots\]
\[\beta_{n-1}a_{n-1}+\alpha_na_n+\beta_na_{n+1}=Ea_n\]
\[\cdots\]
where $\alpha_n=n^2+2n+2, \beta_n=-1$.

So we have the equation $\hat{H}\vec{x}=E\vec{x}$, which can be written as the tri-diagonal form for ode if we cut off the polynomial at the Nth term,
\[\left(\begin{array}{cccccc}
2 & -1 & 0 & 0 & \cdots\cdots  &0\\
-1 & 5 & -1 & 0 & \cdots\cdots & 0\\
0 & -1 & 10 & -1 & \cdots\cdots & 0\\
\vdots&\vdots&\ddots&\ddots&\ddots&\vdots\\
0 & 0 & \cdots & -1 & \alpha_{N-1} & -1\\
0 & 0 & \cdots & 0 &-1& \alpha_N \end{array} \right)\left(\begin{array}{c}
a_0\\a_1\\a_2\\ \vdots\\ a_{N-1}\\ a_N\end{array} \right)=E\left(\begin{array}{c}
a_0\\a_1\\a_2\\ \vdots\\ a_{N-1}\\ a_N\end{array} \right)\]


\item converting this to a continued fraction,
from the equation $\beta_{n-1}a_{n-1}+\alpha_na_n+\beta_na_{n+1}=Ea_n$, divided by $a_n$ get,
\[\beta_{n-1}\frac{a_{n-1}}{a_n}+\alpha_n+\beta_n\frac{a_{n+1}}{a_n}=E,\]
so\[\frac{a_n}{a_{n-1}}=\frac{\beta_{n-1}}{E-\alpha_n-\beta_n\frac{a_{n+1}}{a_n}},\]
apply this equation from n=1 and iterate,
\[\frac{a_1}{a_0}=\frac{\beta_0}{E-\alpha_1-\beta_1\frac{a_2}{a_1}}=\frac{\beta_0}{E-\alpha_1-\frac{\beta_1^2}{E-\alpha_2-\frac{\beta_2^2}{E-\alpha_3-\cdots}}},\]
from the very first equation $\alpha_0a_0+\beta_0a_1=Ea_0$, divided by $a_0$ get,
\[\frac{a_1}{a_0}=\frac{E-\alpha_0}{\beta_0},\]
compare these two equation, we get a equation for E,
\[E=\alpha_0+\frac{\beta_0^2}{E-\alpha_1-\frac{\beta_1^2}{E-\alpha_2-\frac{\beta_2^2}{E-\alpha_3-\cdots}}}\]
\[=2+\frac{1}{E-5-\frac{1}{E-10-\frac{1}{E-17-\cdots}}}\]

\item solving this using Pad\'{e} approximants and bisection root finding.
The continued fraction $\frac{1}{E-5-\frac{1}{E-10-\frac{1}{E-17-\cdots}}}$ has coefficients $a_1=1, a_n=-1\;\;(n\not=1); b_n=E-(n^2+2n+2).$ convert this continued fraction to Pad\'{e} and let $f(E)=2+\frac{1}{E-5-\frac{1}{E-10-\frac{1}{E-17-\cdots}}}-E$, using bisection root finding to find the lowest 8 roots of $f(E)$ for large $N$.
\end{enumerate}
The code:
\begin{verbatim}
program eigenvalue
IMPLICIT NONE
double precision:: a,b,c,d,root
double precision, EXTERNAL:: pade,pA,pB
integer::cond,kl,kr
write (*,*) 'Input the range for root finding A & B:'
read (*,*) A,B
call BISEC(pade,A,B,1d-7,c,d,cond,kl,kr)
print *, cond
if (cond==1) then
   print *, 'Wrong range!'
else if (cond==2) then
   root=c
   print *, 'root=', root
else if (cond==3) then
   root=(a+b)/2.0
   print *, 'root=',root
else if (cond==4) then
   print *, 'Not continued!'
end if
end program

    SUBROUTINE BISEC(pade,A,B,Delta,C,D,Cond,KL,KR)
    double precision, PARAMETER:: Big=1d30
    INTEGER Cond,K,KL,KR,Max
    double precision A,B,C,D,Delta,YA,YB,YC
    double precision, EXTERNAL:: pade,pa,pb
    K=0
    KL=0
    KR=0
    YA=pade(pa,pb,A)
    YB=pade(pa,pb,B)
    D=B-A
    Cond=0
    Max=1+INT((dLOG(D)-dLOG(Delta))/dLOG(2d0))
    IF (YA*YB.GE.0) THEN
        Cond=1
        RETURN
    ENDIF
    DO WHILE ((Cond.EQ.0).AND.(K.LT.Max))
        C=(A+B)/2d0
        YC=pade(pa,pb,C)
            IF (YC.EQ.0d0) THEN
                A=C
                B=C
                Cond=2
            ELSEIF ((YB*YC).GT.0d0) THEN
                B=C
                YB=YC
                KR=KR+1
            ELSE
                A=C
                YA=YC
                KL=KL+1
            ENDIF
        K=K+1
        WRITE(1,*) k,a,c,b
    ENDDO
    D=B-A
    IF (D.LT.Delta) THEN
        IF (Cond.NE.2) Cond=3
        IF (dABS(YA-YB).GT.Big) Cond=4
    ENDIF
    RETURN
   END SUBROUTINE

double precision function pade(pA,pB,E)
IMPLICIT NONE
integer::n=100
double precision,external:: pA,pB
double precision::E
pade=pA(n,E)+pB(n,E)*(2d0-E)
return
end function pade

recursive double precision function pA(n,E) result(Avalue)
IMPLICIT NONE
integer:: n
double precision:: E, Avalue
if (n==0) then
  Avalue=0d0
else if (n==1) then
  Avalue=1d0
else
  Avalue=(E-(n**2+2*n+2))*pA(n-1,E)-pA(n-2,E)
end if 
return
end function pA

recursive double precision function pB(n,E) result(Bvalue)
IMPLICIT NONE
integer:: n
double precision:: E, Bvalue
if (n==0) then
  Bvalue=1d0
else if (n==1) then
  Bvalue=E-5d0
else
  Bvalue=(E-(n**2+2*n+2))*pB(n-1,E)-pB(n-2,E)
end if 
return
end function pB
\end{verbatim}
The bisection subroutine is from \begin{verbatim}http://math.fullerton.edu/mathews/n2003/bisection/BisectionProg/Links/BisectionProg_lnk_5.html.\end{verbatim}  I used mathematica to plot $f(E)$ first so I knew the roots are very close to $\alpha_n$, so for the first 8 $\alpha_n$, they are 2, 5, 10, 17, 26, 37, 50.  So basically I typed in 8 ranges by hand and got the lowest 8 eigenvalues: 
\begin{verbatim}
 Input the range for root finding A & B:
1,2
 root=   1.6867202818393707     
 Input the range for root finding A & B:
5,6
 root=   5.1130088269710541     
 Input the range for root finding A & B:
10,11
 root=   10.057352870702744     
 Input the range for root finding A & B:
17,18
 root=   17.031789809465408     
 Input the range for root finding A & B:
26,27
 root=   26.020212918519974     
 Input the range for root finding A & B:
37,38
 root=   37.013989597558975     
 Input the range for root finding A & B:
50,51   
 root=   50.010257810354233   
\end{verbatim}
I set N=10 because N=100 didn't work for me.


\vskip 1cm

\textbf{Problem3}

Use the quotient-difference algorithm to convert the power series expression of $exp(-x)=sum_{n}c_nx^n$ to continued fraction, where $c_n=\frac{(-1)^n}{n!}$:
\begin{eqnarray*}
e_0^n&=&0\\
q_1^n&=&\frac{c_{n+1}}{c_n}\\
e_m^n&=&q_m^{n+1}-q_m^n+e_{m-1}^{n+1}\\
q_{m+1}{n}&=&\frac{e_{m+1}^n}{e_m^n}q_m^{n+1}
\end{eqnarray*}

I did this first by hand and found the rules, here are the results,
\[e_0^n=0\]
\[q_1^n=-\frac{1}{n+1}\]
\[e_m^n=\frac{m}{(n+2m-1)(n+2m)}\]
\[q_m^n=-\frac{n+m-1}{(n+2m-2)(n+2m-1)}\]

so here $b_n=1$, $a_1=1, a_n=q_{n/2}^0$ (n is even), $a_n=e_{(n-1)/2}^0$ (n is odd).

Then I tried two recursive function $e$ and $q$ to let them call each other to get the value of $e$s and $q$s but failed so I used two arrays $e$ and $q$ to do the iteration and set another array to locate $a_n$.  Here are the results of  $e$s, $q$s and $a_n$ from the program, which do correspond to the results I got by hand.
\begin{verbatim}
            m         n
           1           0  q= -1.00000000000000000     
           1           1  q= -0.50000000000000000     
           1           2  q= -0.33333333333333331     
           1           3  q= -0.25000000000000000     
           1           4  q= -0.20000000000000001     
           1           5  q= -0.16666666666666666     
           1           6  q= -0.14285714285714285     
           1           7  q= -0.12500000000000000     
           1           8  q= -0.11111111111111110     
           1           9  q= -0.10000000000000001     
           1          10  q= -9.09090909090909116E-002
           1          11  q= -8.33333333333333287E-002
           1          12  q= -7.69230769230769273E-002
           1          13  q= -7.14285714285714246E-002
           1          14  q= -6.66666666666666657E-002
           1          15  q= -6.25000000000000000E-002
           1          16  q= -5.88235294117647051E-002
           1          17  q= -5.55555555555555525E-002
           1          18  q= -5.26315789473684181E-002
           1          19  q= -5.00000000000000028E-002
           1           0  e=  0.50000000000000000     
           1           1  e=  0.16666666666666669     
           1           2  e=  8.33333333333333148E-002
           1           3  e=  4.99999999999999889E-002
           1           4  e=  3.33333333333333537E-002
           1           5  e=  2.38095238095238082E-002
           1           6  e=  1.78571428571428492E-002
           1           7  e=  1.38888888888888951E-002
           1           8  e=  1.11111111111110994E-002
           1           9  e=  9.09090909090909394E-003
           1          10  e=  7.57575757575758291E-003
           1          11  e=  6.41025641025640136E-003
           1          12  e=  5.49450549450550274E-003
           1          13  e=  4.76190476190475886E-003
           1          14  e=  4.16666666666666574E-003
           1          15  e=  3.67647058823529493E-003
           1          16  e=  3.26797385620915259E-003
           1          17  e=  2.92397660818713434E-003
           1          18  e=  2.63157894736841536E-003
           2           0  q= -0.16666666666666669     
           2           1  q= -0.16666666666666660     
           2           2  q= -0.14999999999999999     
           2           3  q= -0.13333333333333344     
           2           4  q= -0.11904761904761897     
           2           5  q= -0.10714285714285710     
           2           6  q= -9.72222222222223070E-002
           2           7  q= -8.88888888888887535E-002
           2           8  q= -8.18181818181819426E-002
           2           9  q= -7.57575757575758013E-002
           2          10  q= -7.05128205128203456E-002
           2          11  q= -6.59340659340661300E-002
           2          12  q= -6.19047619047617681E-002
           2          13  q= -5.83333333333333551E-002
           2          14  q= -5.51470588235294379E-002
           2          15  q= -5.22875816993464276E-002
           2          16  q= -4.97076023391812491E-002
           2          17  q= -4.73684210526314764E-002
           2           0  e=  0.16666666666666677     
           2           1  e=  9.99999999999999223E-002
           2           2  e=  6.66666666666665408E-002
           2           3  e=  4.76190476190478246E-002
           2           4  e=  3.57142857142856845E-002
           2           5  e=  2.77777777777776375E-002
           2           6  e=  2.22222222222224486E-002
           2           7  e=  1.81818181818179103E-002
           2           8  e=  1.51515151515152352E-002
           2           9  e=  1.28205128205130386E-002
           2          10  e=  1.09890109890106169E-002
           2          11  e=  9.52380952380986467E-003
           2          12  e=  8.33333333333317189E-003
           2          13  e=  7.35294117647058293E-003
           2          14  e=  6.53594771241830519E-003
           2          15  e=  5.84795321637433113E-003
           2          16  e=  5.26315789473690704E-003
           3           0  q= -9.99999999999998251E-002
           3           1  q= -9.99999999999998945E-002
           3           2  q= -9.52380952380959128E-002
           3           3  q= -8.92857142857137603E-002
           3           4  q= -8.33333333333329401E-002
           3           5  q= -7.77777777777790280E-002
           3           6  q= -7.27272727272707808E-002
           3           7  q= -6.81818181818196756E-002
           3           8  q= -6.41025641025648740E-002
           3           9  q= -6.04395604395572272E-002
           3          10  q= -5.71428571428612905E-002
           3          11  q= -5.41666666666635599E-002
           3          12  q= -5.14705882352950936E-002
           3          13  q= -4.90196078431373514E-002
           3          14  q= -4.67836257309946352E-002
           3          15  q= -4.47368421052631998E-002
           3           0  e=  9.99999999999998529E-002
           3           1  e=  7.14285714285705226E-002
           3           2  e=  5.35714285714299771E-002
           3           3  e=  4.16666666666665048E-002
           3           4  e=  3.33333333333315496E-002
           3           5  e=  2.72727272727306957E-002
           3           6  e=  2.27272727272690156E-002
           3           7  e=  1.92307692307700367E-002
           3           8  e=  1.64835164835206854E-002
           3           9  e=  1.42857142857065536E-002
           3          10  e=  1.25000000000075953E-002
           3          11  e=  1.10294117647016382E-002
           3          12  e=  9.80392156862832514E-003
           3          13  e=  8.77192982456102138E-003
           3          14  e=  7.89473684210576648E-003
           4           0  q= -7.14285714285705503E-002
           4           1  q= -7.14285714285747136E-002
           4           2  q= -6.94444444444419495E-002
           4           3  q= -6.66666666666630436E-002
           4           4  q= -6.36363636363760504E-002
           4           5  q= -6.06060606060414842E-002
           4           6  q= -5.76923076923207961E-002
           4           7  q= -5.49450549450673092E-002
           4           8  q= -5.23809523809078970E-002
           4           9  q= -5.00000000000610720E-002
           4          10  q= -4.77941176470086521E-002
           4          11  q= -4.57516339869506530E-002
           4          12  q= -4.38596491228012836E-002
           4          13  q= -4.21052631578996903E-002
           4           0  e=  7.14285714285663592E-002
           4           1  e=  5.55555555555627412E-002
           4           2  e=  4.44444444444454106E-002
           4           3  e=  3.63636363636185428E-002
           4           4  e=  3.03030303030652620E-002
           4           5  e=  2.56410256409897036E-002
           4           6  e=  2.19780219780235236E-002
           4           7  e=  1.90476190476800977E-002
           4           8  e=  1.66666666665533786E-002
           4           9  e=  1.47058823530600152E-002
           4          10  e=  1.30718954247596372E-002
           4          11  e=  1.16959064327776946E-002
           4          12  e=  1.05263157894626147E-002
           5           0  q= -5.55555555555692360E-002
           5           1  q= -5.55555555555475797E-002
           5           2  q= -5.45454545454236647E-002
           5           3  q= -5.30303030304005371E-002
           5           4  q= -5.12820512819040716E-002
           5           5  q= -4.94505494506334645E-002
           5           6  q= -4.76190476192076134E-002
           5           7  q= -4.58333333328359641E-002
           5           8  q= -4.41176470595338113E-002
           5           9  q= -4.24836601300809108E-002
           5          10  q= -4.09356725149831249E-002
           5          11  q= -3.94736842103812421E-002
           5           0  e=  5.55555555555843975E-002
           5           1  e=  4.54545454545693256E-002
           5           2  e=  3.78787878786416704E-002
           5           3  e=  3.20512820515617275E-002
           5           4  e=  2.74725274722603108E-002
           5           5  e=  2.38095238094493747E-002
           5           6  e=  2.08333333340517471E-002
           5           7  e=  1.83823529398555313E-002
           5           8  e=  1.63398692825129158E-002
           5           9  e=  1.46198830398574231E-002
           5          10  e=  1.31578947373795774E-002
           6           0  q= -4.54545454545392039E-002
           6           1  q= -4.54545454543203997E-002
           6           2  q= -4.48717948724421251E-002
           6           3  q= -4.39560439551067578E-002
           6           4  q= -4.28571428574984620E-002
           6           5  q= -4.16666666683737502E-002
           6           6  q= -4.04411764658487508E-002
           6           7  q= -3.92156862814803595E-002
           6           8  q= -3.80116958996453286E-002
           6           9  q= -3.68421052677069191E-002
           6           0  e=  4.54545454547881297E-002
           6           1  e=  3.84615384605199451E-002
           6           2  e=  3.29670329688970948E-002
           6           3  e=  2.85714285698686066E-002
           6           4  e=  2.49999999985740864E-002
           6           5  e=  2.20588235365767465E-002
           6           6  e=  1.96078431242239226E-002
           6           7  e=  1.75438596643479466E-002
           6           8  e=  1.57894736717958326E-002
           7           0  q= -3.84615384601241783E-002
           7           1  q= -3.84615384652865835E-002
           7           2  q= -3.80952380901918020E-002
           7           3  q= -3.75000000002197395E-002
           7           4  q= -3.67647058978977648E-002
           7           5  q= -3.59477123785213806E-002
           7           6  q= -3.50877193582513000E-002
           7           7  q= -3.42105262530932736E-002
           7           0  e=  3.84615384553575398E-002
           7           1  e=  3.33333333439918764E-002
           7           2  e=  2.91666666598406690E-002
           7           3  e=  2.57352941008960612E-002
           7           4  e=  2.28758170559531307E-002
           7           5  e=  2.04678361444940032E-002
           7           6  e=  1.84210527695059731E-002
           8           0  q= -3.33333333525970485E-002
           8           1  q= -3.33333333104581450E-002
           8           2  q= -3.30882352805183175E-002
           8           3  q= -3.26797386864503409E-002
           8           4  q= -3.21637423893237281E-002
           8           5  q= -3.15789478329313797E-002
           8           0  e=  3.33333333861307798E-002
           8           1  e=  2.94117646897804966E-002
           8           2  e=  2.61437906949640378E-002
           8           3  e=  2.33918133530797434E-002
           8           4  e=  2.10526307008863517E-002
           9           0  q= -2.94117646230105251E-002
           9           1  q= -2.94117645358481054E-002
           9           2  q= -2.92397669756481078E-002
           9           3  q= -2.89473663396739612E-002
           9           0  e=  2.94117647769429162E-002
           9           1  e=  2.63157882551640354E-002
           9           2  e=  2.36842139890538900E-002
          10           0  q= -2.63157880394476358E-002
          10           1  q= -2.63157953440906409E-002
          10           0  e=  2.63157809505210304E-002
a1=  1.00000000000000000     
a2=  1.00000000000000000     
a3= -0.50000000000000000     
a4=  0.16666666666666669     
a5= -0.16666666666666677     
a6=  9.99999999999998251E-002
a7= -9.99999999999998529E-002
a8=  7.14285714285705503E-002
a9= -7.14285714285663592E-002
a10=  5.55555555555692360E-002
a11= -5.55555555555843975E-002
a12=  4.54545454545392039E-002
a13= -4.54545454547881297E-002
a14=  3.84615384601241783E-002
a15= -3.84615384553575398E-002
a16=  3.33333333525970485E-002
a17= -3.33333333861307798E-002
a18=  2.94117646230105251E-002
a19= -2.94117647769429162E-002
a20=  2.63157880394476358E-002
a21= -2.63157809505210304E-002
\end{verbatim}

The code:
\begin{verbatim}
program quotientDifference
integer,parameter:: NUM=30
integer::n,m,i,j,flag,pn,step
double precision::x,xmax,dx,error
double precision::pade,Avalue,Bvalue,Avalue_1,Avalue_2,Bvalue_1,Bvalue_2
double precision, dimension(0:NUM-1):: e,q
double precision, dimension(1:NUM):: a
!STEP 1, get an for cotinued fraction using quotient-difference algorithm
a(1)=1d0
!initial value for es
flag=1!flag is the index of array a, and also the total column of e and q
m=0!m is the lower index of e and p
e=0d0
!initial value for qs
flag=flag+1
m=m+1
do n=0,NUM-flag
   q(n)=-1d0/(dble(n)+1d0)
print *, m,n,'q=', q(n)
end do
a(flag)=-q(0)
!calculate es and qs and get the value of as
do i=3,NUM,2
   flag=i
   do n=0,NUM-flag
      e(n)=q(n+1)-q(n)+e(n+1) 
   print *, m,n, 'e=',e(n)
   end do
   a(flag)=-e(0)
   m=m+1
   flag=flag+1
   do n=0,NUM-flag
      q(n)=e(n+1)/e(n)*q(n+1)
   print *, m,n, 'q=',q(n)
   end do  
   a(flag)=-q(0)
end do
do i=1, NUM
   print*, a(i)
end do
!STEP 2, evaluate using pade
print *, 'enter the value of N for Pade:'
read *, pn
x=0d0
xmax=40d0
step=1000
dx=(xmax-0d0)/step
do i=0,step
   Avalue_1=0d0
   Avalue_2=1d0
   Bvalue_1=1d0
   Bvalue_2=0d0  
   do j=1,pn
      if (j==1) then
         Avalue=Avalue_1+a(j)*Avalue_2
         Bvalue=Bvalue_1+a(j)*Bvalue_2
      else
         Avalue=Avalue_1+a(j)*x*Avalue_2
         Bvalue=Bvalue_1+a(j)*x*Bvalue_2
      end if
      Avalue_2=Avalue_1
      Avalue_1=Avalue
      Bvalue_2=Bvalue_1
      Bvalue_1=Bvalue
   end do
   pade=Avalue/Bvalue
   error=dabs(pade-dexp(-x))/dexp(-x)
   write(1,*) x,dexp(-x),pade,error
   x=x+dx
end do
end program quotientDifference
\end{verbatim}

\begin{figure} [ht]
\includegraphics[width=1.0\textwidth]{4_3}
\end{figure}
From the relative error  plot,  it can be seen that both 5th and 10th approximants are good estimation for $exp(-x)$ for small $x$, but bad for large $x$.  The 10th approximants are about at least 2 order better than the 5th approximants for large $x$.  And there are some small oscillation for 10th approximants at the very beginning.
 \vskip 1cm
  
\textbf{Problem4}
  
  I set $A_n=P_{nn}$ here and fix the $N$ to be 2 while varying n from 1 to 30 because this method kind of makes sense to me.  Both of the Pad\'{e} and Richardson jumps a little bit at the beginning and then converge to a quite good value.  But I didn't see any obvious improvement from Pad\'{e} to Richardson.
  
The code:
\begin{verbatim}
program main
 IMPLICIT NONE
 DOUBLEPRECISION :: richard
 INTEGER :: k, nlow, ncap
 ncap = 2!ncap is the number of Pade temrs being grouped together to form Richardson transformation. The N index in the expression of Q0 
 WRITE(1,*) "N for Richardson is", ncap
 WRITE(1,*) "n,       Pade,       Richardson,       exp(-10)"
do nlow=1,20 !nlow is the N index of P_NN
!do Richardson series
   richard = 0D0
  DO k = 0, ncap 
   richard = richard + pade(2*(k+nlow)+1) *dble(nlow+k)**ncap * (-1D0)**(k+ncap)/dble( factorial(k)*factorial(ncap-k) )
  ENDDO 
  WRITE(1, '(I2, 1F20.15, 2ES25.14)') nlow, pade(2*nlow+1), richard, DEXP(-10D0)
end do
contains
double precision function pade(pn)
integer,parameter:: NUM=60
integer::n,m,i,flag,pn
double precision::x=10d0,Avalue,Bvalue,Avalue_1,Avalue_2,Bvalue_1,Bvalue_2
double precision, dimension(0:NUM-1):: e,q
double precision, dimension(1:NUM):: a
!STEP 1, get an for cotinued fraction using quotient-difference algorithm
a(1)=1d0
!initial value for es
flag=1!flag is the index of array a, and also the total column of e and q
m=0!m is the lower index of e and p
e=0d0
!initial value for qs
flag=flag+1
m=m+1
do n=0,NUM-flag
   q(n)=-1d0/(dble(n)+1d0)
end do
a(flag)=-q(0)
!calculate es and qs and get the value of as
do i=3,NUM,2
   flag=i
   do n=0,NUM-flag
      e(n)=q(n+1)-q(n)+e(n+1) 
   end do
   a(flag)=-e(0)
   m=m+1
   flag=flag+1
   do n=0,NUM-flag
      q(n)=e(n+1)/e(n)*q(n+1)
   end do  
   a(flag)=-q(0)
end do
!STEP 2, evaluate using pade
   Avalue_1=0d0
   Avalue_2=1d0
   Bvalue_1=1d0
   Bvalue_2=0d0  
   do i=1,pn
      if (i==1) then
         Avalue=Avalue_1+a(i)*Avalue_2
         Bvalue=Bvalue_1+a(i)*Bvalue_2
      else
         Avalue=Avalue_1+a(i)*x*Avalue_2
         Bvalue=Bvalue_1+a(i)*x*Bvalue_2
      end if
      Avalue_2=Avalue_1
      Avalue_1=Avalue
      Bvalue_2=Bvalue_1
      Bvalue_1=Bvalue
   end do
   pade=Avalue/Bvalue
end function pade
integer FUNCTION factorial(nn)
  IMPLICIT NONE
  INTEGER :: iter,nn,fact
  fact=1
  DO iter = 1, nn
   fact = fact*DBLE(iter)
  ENDDO
   factorial = fact
   RETURN
 END FUNCTION
end program
\end{verbatim}
  
\begin{figure} [ht]
\includegraphics[width=1.0\textwidth]{4_4}
\end{figure}
  
  
  
\end{document}