\documentclass{article}
\usepackage{epsf,epsfig,graphicx}
\topmargin=-15mm \oddsidemargin=-5mm \textwidth=170mm \textheight=245mm \evensidemargin=-5mm
\title{HW10 Finite Temperature 2-d Ising Model}
\author{Fei Ding}
\begin{document}
\maketitle
\vskip 1cm
I set a 60$\times$ 60 checkerboard and let the sweep go 4000 times.  Figure 1 shows autocorrelation, one action means one sweep.  I pike up $\tau=12$, which has the smallest autocorrelation.  So I would pick up data points every 12 sweeps while calculating average magnetization in part (ii).

  \begin{figure}[ht]
   \begin{center}
   \includegraphics{autocor.eps}
   \caption{autocorrelation vs. $\tau$}
   \end{center}
  \end{figure}

\begin{enumerate}
\item[(i)]
The magnetization for each sweep is calculated as,
\[M=\frac{1}{L^2}\sum_{\mathtt{all \; spin\; on\; array\; i}} \sigma_i \propto (n_{\uparrow}-n_{\downarrow})\]
I plot data for every 12 sweep.
The magnetization vs. sweep for $kT=2.0$ and $kT=2.5$ were shown on figure 2 with different initial conditions(random start and all spins up start for both cases).  It shows that the initial start didn't affect the result.  Under the same temperature, the magnetization would converge to the same value.  For $kT=2.0$, the particles tend to spin up, and it would finally get a magnetization around 0.9, while for $kT=2.5$, the particles tend to have random spins, so the magnetization oscillates around zero.
 \begin{figure}[ht]
   \begin{center}
   {\includegraphics{thermal.eps}}
   \caption{Magnetization vs. sweep}
   \end{center}
  \end{figure}
 
\item[(ii)]
 The average magnetization and the error were calculated as,
\[\left< M \right> =\frac{1}{\mathtt{number of sweep}} \sum_{\mathtt{sweep}\; i} M_i\]
Every 12 sweeps were counted as one data point, correspondingly, the number of sweep should also reduce to sweep/12 to do the average.  The curve and fitting was shown on figure 3.  The fit result is $kT_c=2.33$
 \begin{figure}[ht]
   \begin{center}
   {\includegraphics{ktvary.eps}}
   \caption{Average Magnetization vs. kT}
   \end{center}
  \end{figure}

\item[(iii)]
For the spin-spin correlation, I only calculated the up, down, left, right neighbour correlation.  Divide the region into four parts to get error estimation.
For $kT=2.0$ case, I used straight line to fit the functions, but the fit is a little bit odd.
 \begin{figure}[ht]
   \begin{center}
   {\includegraphics{spincor.eps}}
   \caption{log Spin-Spin correlation function for kT=2.0}
   \end{center}
  \end{figure}
  
 \begin{figure}[ht]
   \begin{center}
   {\includegraphics{spincorhigh.eps}}
   \caption{Spin-Spin correlation function for kT=2.5}
   \end{center}
  \end{figure}


 \begin{figure}[ht]
   \begin{center}
   {\includegraphics{spincorTc.eps}}
   \caption{Spin-Spin correlation function for kTc=2.33}
   \end{center}
  \end{figure}
\end{enumerate}
\newpage
\appendix
\section{Codes}
\begin{verbatim}
PROGRAM main

IMPLICIT NONE
DOUBLEPRECISION :: pi
INTEGER :: seed
DOUBLEPRECISION :: RAND
INTEGER, PARAMETER :: sweepMax = 4000
DOUBLEPRECISION, DIMENSION(sweepMax) :: netM  !netM at the ith sweep
INTEGER, PARAMETER :: n = 60   !50*50 spins total
INTEGER, DIMENSION(0:n+1,0:n+1) :: spin  
INTEGER :: Nup, Ndown
DOUBLEPRECISION :: kT
DOUBLEPRECISION :: sum_netM,avgM 
DOUBLEPRECISION :: deltaE  !change in energy
DOUBLEPRECISION :: a,random  ! the random number to determine hopping or not
INTEGER :: i,j,sweep

!autocorrelation
INTEGER :: t !tao for autocorrelation
DOUBLEPRECISION :: MiMit,Mi2,Mi,ct !cross term avg,square avg,avg and autocorrelation

!pin-spin correlation
INTEGER :: r !distance
DOUBLEPRECISION, DIMENSION(4) :: spincrossavg, spinsquareavg, spinavg !I'm dividing the area to 4 pieces, so 4 data points
DOUBLEPRECISION, DIMENSION(4) :: spincor
DOUBLEPRECISION :: spincorsum, spincorsquaresum,correlation,correlation2


seed = 918172
CALL SRAND(seed)

kT = 2.33D0
!Initializin
DO i = 1, n
	DO j = 1, n
	    a=RAND()
		IF ( a > 0.5D0 ) THEN
		    spin(i,j)=1
			Nup = Nup + 1
		ELSE if (a <=0.5d0) then
		    spin(i,j)=-1
			Ndown = Ndown + 1
		ENDIF
		WRITE(11,*) i, j, spin(i,j)
	ENDDO 
ENDDO 




DO i = 1, n
	DO j = 1, n
		IF (i==1)  THEN   !PBC
			spin(i-1,j) = spin(n,j)
		ELSEIF (i==n)  THEN
			spin(i+1,j) = spin(1,j)
		ELSE
			CONTINUE
		ENDIF
		IF (j==1)  THEN
			spin(i,j-1) = spin(i,n) 
		ELSEIF (j==n)  THEN
			spin(i,j+1) = spin(i,1)
		ELSE
			CONTINUE
		ENDIF
	!	hamiltonian = hamiltonian + spin(i,j)*( spin(i+1,j) + spin(i-1,j) + spin(i,j-1) + spin(i,j+1) )
	ENDDO
ENDDO
!hamiltonian = -0.5D0 * hamiltonian !count pair for once

!all up start
!spin=1   
!ndown = 0
!nup = n**2
print *,'nup=',nup,'ndown=',ndown

sum_netM = 0D0
DO sweep = 1, sweepMax
!change configureation, flipping one by one
	DO i = 1, n
		DO j = 1, n
			IF (i==1)  THEN   !reset PBC for each sweep
				spin(i-1,j) = spin(n,j)
			ELSEIF (i==n)  THEN
				spin(i+1,j) = spin(1,j)
			ELSE
				CONTINUE
			ENDIF
			IF (j==1)  THEN
				spin(i,j-1) = spin(i,n) 
			ELSEIF (j==n)  THEN
				spin(i,j+1) = spin(i,1)
			ELSE
				CONTINUE
			ENDIF
			deltaE = 2D0 * spin(i,j) * ( spin(i+1,j) + spin(i-1,j) + spin(i,j-1) + spin(i,j+1) )    !check
	        !spin:1 to -1 or -1 to 1 	
			IF (deltaE < 0D0) THEN   !accept the flip
				Nup = Nup - spin(i,j)
				Ndown = Ndown + spin(i,j)
				spin(i,j) = -spin(i,j)
			ELSE                   
				random = RAND()
				IF (random < DEXP(-deltaE/kt)) THEN  !accept with some probability
					Nup = Nup - spin(i,j)
					Ndown = Ndown + spin(i,j)
					spin(i,j) = -spin(i,j) 
				ELSE
					spin(i,j) = spin(i,j)   !do nothing
				ENDIF
			ENDIF
		ENDDO 
	ENDDO 
	netM(sweep) = DBLE(Nup - Ndown)/DBLE(n**2)
	IF (MOD(sweep,12)==0) THEN
		sum_netM = sum_netM + netM(sweep)
		avgM = sum_netM/dble(sweep/12)
		WRITE(23,*) sweep/12, netM(sweep)
	ENDIF

IF (mod(sweep, 100)==0)   THEN
!WRITE(*,*) hamil, netM(sweep), avgM
PRINT*, nup, ndown, nup+ndown
ENDIF

ENDDO !sweep


!=====autocorrelation=================================
DO t = 0, 50
	MiMit = 0D0
	Mi2 = 0D0
	Mi = 0D0

		DO sweep = 2001,sweepMax-50
			MiMit = MiMit + netM(sweep) * netM(sweep+t)
			Mi2 = Mi2 + netM(sweep)**2
			Mi = Mi + netM(sweep)
		ENDDO
		MiMit = MiMit/DBLE(sweepMax-50-2000)
		Mi2 = Mi2/DBLE(sweepMax-50-2000)
		Mi = Mi/DBLE(sweepMax-50-2000)

		ct = ( MiMit - Mi**2 )/( Mi2 - Mi**2 )

	WRITE(12,*) t, ct

ENDDO


!=====spin-spin correlation======================================
DO r = 0, n/4
	spincrossavg = 0D0
	spinsquareavg = 0D0
	spinavg = 0D0

	DO i = n/4+1, n/2  !upperleft coner
		DO j = n/2+1, n*3/4    
			spincrossavg(1) = spincrossavg(1) + DBLE(spin(i,j)* ( spin(i+r,j) + spin(i-r,j) + spin(i,j+r) + spin(i,j-r))/4D0 )
			spinsquareavg(1) = spinsquareavg(1) + DBLE(spin(i,j)**2)
			spinavg(1) = spinavg(1) + DBLE(spin(i,j))	
		ENDDO 
	ENDDO 
	spincrossavg(1) = spincrossavg(1)/DBLE((n/4)**2)
	spinsquareavg(1) = spinsquareavg(1)/DBLE((n/4)**2)
	spinavg(1) = spinavg(1)/DBLE((n/4)**2)

	DO i = n/4+1, n/2    !lowerleft corner
		DO j = n/4+1, n/2
			spincrossavg(2) = spincrossavg(2) + DBLE(spin(i,j)* ( spin(i+r,j) + spin(i-r,j) + spin(i,j+r) + spin(i,j-r))/4D0 )
			spinsquareavg(2) = spinsquareavg(2) + DBLE(spin(i,j)**2)
			spinavg(2) = spinavg(2) + DBLE(spin(i,j))	
		ENDDO 
	ENDDO 
	spincrossavg(2) = spincrossavg(2)/DBLE((n/4)**2)
	spinsquareavg(2) = spinsquareavg(2)/DBLE((n/4)**2)
	spinavg(2) = spinavg(2)/DBLE((n/4)**2)

	DO i = n/2+1, n*3/4    !upperright corner
		DO j = n/2+1, n*3/4
			spincrossavg(3) = spincrossavg(3) + DBLE(spin(i,j)* ( spin(i+r,j) + spin(i-r,j) + spin(i,j+r) + spin(i,j-r))/4D0 )
			spinsquareavg(3) = spinsquareavg(3) + DBLE(spin(i,j)**2)
			spinavg(3) = spinavg(3) + DBLE(spin(i,j))	
		ENDDO 
	ENDDO 
	spincrossavg(3) = spincrossavg(3)/DBLE((n/4)**2)
	spinsquareavg(3) = spinsquareavg(3)/DBLE((n/4)**2)
	spinavg(3) = spinavg(3)/DBLE((n/4)**2)

	DO i = n/2+1, n*3/4    !lowerright coner
		DO j = n/4+1, n*3/4
			spincrossavg(4) = spincrossavg(4) + DBLE(spin(i,j)* ( spin(i+r,j) + spin(i-r,j) + spin(i,j+r) + spin(i,j-r))/4D0 )
			spinsquareavg(4) = spinsquareavg(4) + DBLE(spin(i,j)**2)
			spinavg(4) = spinavg(4) + DBLE(spin(i,j))	
		ENDDO 
	ENDDO 
	spincrossavg(4) = spincrossavg(4)/DBLE((n/4)**2)
	spinsquareavg(4) = spinsquareavg(4)/DBLE((n/4)**2)
	spinavg(4) = spinavg(4)/DBLE((n/4)**2)

	spincorsum = 0D0
	spincorsquaresum = 0D0
    DO i = 1,4
		spincor(i) = (spincrossavg(i) - spinavg(i)**2)/(spinsquareavg(i) - spinavg(i)**2)
		spincorsum = spincorsum + spincor(i)
		spincorsquaresum = spincorsquaresum + spincor(i)**2
	ENDDO 
    correlation=spincorsum/4d0
	correlation2=spincorsquaresum/4d0

	WRITE(34,'(I2, 4ES20.10)') r, (correlation), DSQRT( correlation2 - correlation**2 )/DSQRT(3d0)

ENDDO 

DO i = 1, n
	DO j = 1, n
		WRITE(11,*) i, j, spin(i,j)
	ENDDO 
ENDDO 
END PROGRAM

\end{verbatim}
\end{document}